\documentclass[a4paper,fleqn,usenatbib]{mnras}

%=========================================================================
\usepackage{amsmath} 
\usepackage{amssymb} 
\usepackage{multirow}

\usepackage{graphicx}
\usepackage{grffile}
\usepackage[dvips]{epsfig}
\usepackage{epsfig}  
\usepackage{color}
\usepackage{caption}
\usepackage{hyperref}
\usepackage{bm}
%Non reposionated tables




%=========================================================================
%		INTERNAL MACROS
%=========================================================================
\def\be{\begin{equation}}
\def\ee{\end{equation}}
\def\ba{\begin{eqnarray}}
\def\ea{\end{eqnarray}}

% To highlight comments 
\definecolor{red}{rgb}{1,0.0,0.0}
\newcommand{\red}{\color{red}}
\definecolor{darkgreen}{rgb}{0.0,0.5,0.0}
\newcommand{\SRK}[1]{\textcolor{darkgreen}{\bf SRK: \textit{#1}}}
\newcommand{\SRKED}[1]{\textcolor{darkgreen}{\bf #1}}
\newcommand{\before}[1]{\textcolor{red}{ #1}}
\newcommand{\after}[1]{\textcolor{darkgreen}{ #1}}
\newcommand{\hs}{{\hspace{1mm}}}  
\newcommand{\tol}{Tololo 1214-277}
\newcommand{\rank}{\texttt{Ranked}}
\newcommand{\boot}{\texttt{Bootstrapped}}
\newcommand{\rand}{\texttt{Random}}
\newcommand{\HI}{{\text{H\MakeUppercase{\romannumeral 1}}} }
\newcommand{\HII}{{\text{H\MakeUppercase{\romannumeral 2}}} }
\newcommand{\lya}{\ifmmode{{\rm Ly}\alpha}\else Ly$\alpha$\ \fi}
\newcommand{\cm}{\ifmmode{{\rm cm}}\else cm\fi}
\newcommand{\ccm}{\,\mathrm{cm}^{-3}}
\newcommand{\ergps}{\,{\rm erg}\,{\rm s}\ifmmode{}^{-1}\else ${}^{-1}$\fi}
\newcommand{\Mpch}{\,{\rm Mpc}\,\ifmmode h^{-1}\else $h^{-1}$\fi}
\newcommand{\dd}{\mathrm{d}}
\newcommand{\vek}[1]{\bm{#1}}
\newcommand{\hb}{H$\beta$}
\newcommand{\ha}{H$\alpha$}
\newcommand{\oiii}{[OIII]}
\newcommand{\oii}{[OII]}
\newcommand{\nii}{[NII]}
\newcommand{\esca}{erg cm$^{-2}$ s$^{-1}$ \AA$^{-1}$}
\newcommand{\esc}{erg cm$^{-2}$ s$^{-1}$}
\newcommand{\es}{erg s$^{-1}$}
\newcommand{\esa}{erg s$^{-1}$}
\newcommand{\kms}{\ifmmode\mathrm{km\ s}^{-1}\else km s$^{-1}$\fi}
\newcommand{\hMsun}{{\ifmmode{h^{-1}{\rm{M_{\odot}}}}\else{$h^{-1}{\rm{M_{\odot}}}$}\fi}}
\newcommand{\Msun}{{\ifmmode{{\rm{M_{\odot}}}}\else{${\rm{M_{\odot}}}$}\fi}}

\newcommand{\jefr}[1]{\textcolor{darkgreen}{\bf JEFR: \textit{#1}}}

\begin{document}

%=========================================================================
%		FRONT MATTER
%=========================================================================
\title[LG satellites distribution asphericity]{We are not the 99 percent: quantifying
  asphericity in the distribution of Local Group satellites}
\author[J.E. Forero-Romero \& V. Arias]
{Jaime E. Forero-Romero $^{1}$ \thanks{je.forero@uniandes.edu.co},
Ver\'onica Arias$^1$\\
%%
$^1$ Departamento de F\'isica, Universidad de los Andes, Cra. 1
  No. 18A-10 Edificio Ip, CP 111711, Bogot\'a, Colombia \\
}

\maketitle

\begin{abstract}
In this paper we quantify the asphericity in the spatial distribution of the
brightest satellites around the Milky Way (MW) and M31, the dominant
galaxies in the Local Group (LG).
We use the Illustris-1 and ELVIS simulations to quantify asphericity
in the Lambda Cold Dark Matter paradigm.  
We build explicit probability distributions of
finding an aspherical satellite distribution, allowing for an easy
reproducibility of the results presented here and a standarized method
to compare probability distributions derived from different
simulations. 
We estimate that only $0.05\%$ to $0.25\%$ of the pairs with isolation
and dynamical characteristics similar to the LG are expected to have
satellite distributions with the same degree of deviation from
sphericity.
This makes the LG a $3\sigma$ outlier, with all the weight lying on
the extremly planar MW satellite distribution.

\end{abstract}

\begin{keywords}Galaxies: halos --- Galaxies: high-redshift --- Galaxies: statistics
--- Dark Matter --- Methods: numerical 
\end{keywords}

\section{Introduction}

The presence of a Vast Polar Structure of satellites around the Milky
Way has been stablished in the last decade.
This structure can be easily described by global quantities of the
brightest


The question whether this configuration is atypical in the observable
universe cannot be proberly addressed given the current limits in
observations. 
So far it has only been possible to contraint that $5\%$ of the
Milky-Way like galaxies in the Universe have two bright satellite
galaxies as is the case of the Milky Way and its Large and Small
Magellanic Clouds. 

The closest system that could help us in providing a baseline for
comparison is the Andromeda galaxy, M31. 
Its close enough that an observational study of its satellite
population is possible, with the great advantage of having a
perpective from a distance and not from within, as is the case for our
Galaxy. 
These studies also found a planar satellite
distributions.
This time not in the whole distribution, but in a subset of $15$
satellites out of an homogeneous sample of $27$ satellites. 


This has prompted a discussion on whether this satellite distributions
are easily found in structure formation simulations done in the
current dominant paradigm for galaxy formation.
This paradigm, based on a Cold Dark Matter cosmology in
a expanding universe described by General Relativity with a
Cosmological Constant, is the so-called Lambda Cold Dark Matter
(LCDM).

The consensus is that these structures are hard to find in
simulations. 
Some studies have focused on the study of individual dark matter halos
that could host galaxies like the MW or M31, while some other studies
have tried to simulate the formation of galaxy pairs resembling both
the MW and M31.
The transition from studying individual halos to pairs has been
motivated by the hypotesis hat the location of the Local Group in
the cosmic web should play a determinant role in building up planar
satellite distributions through preferential alignments and accretions
histories of galaxies.

Studies of the Local Group in an explicit cosmological context have
also renewed the interest on performing constrained simulations that
could reproduce the observed large scale structure of the Universe,
This has also motivated the study of the two dominant galaxies in the 
Local Group as a pair in a cosmological context finding that the LG
itself has relatively uncommon kinematic configuration in the context
of LCDM and a strong preference to lie along filaments.


The aim of our work is to put together some of these pieces in a 
simple framework to quantify how atypical is the satellite
distribution found in the Local Group.
Some of the explicit characteristics we want to keep in our treatment
of the problem are the (i) cosmological context, (ii) the LG as a pair
of galaxies and (iii) a simple and robust description for the
satellites. 

The first condition constrains the kind of simulations we use. 
We either results from fully cosmological simulations or resimulations
of cosmological sub-volumes. 
The second condition constraints the kind of samples we want to
use. We build explicit samples of pairs that resemble the kinematic
structure of the MW and M31. 
The third condition motivates to only use the inertia tensor as a
description for the satellites positions, while droping any
information that might come from 3D velocities (high uncertainties) or
involved algorithms for plane fitting.  

Finally, to compute the odds of finding the LG satellite distributions 
in LCDM simulations, we do not make a direct search for the exact
observed values into simulations because, as we show in Section, the
LG is itself rare and we do not have enough simulated systems to
perform such a brute force solution. 
We also avoid this direct comparison for a second reason. 
We do not want to implicitly use either LCDM or observations as what
is considered normal or standard and compute the deviations from that
point. 
Our approach is different. 
We set as a point of reference an spherical satellite
distribution to quantify how both simulations and observations
deviate from the spherical distribution.
Simulations help us to construct an explicit probability distribution
for deviations from sphericity in LCDM.
Then we compute the odds of finding n satellite distributions that
deviates from sphericity as much as the LG does. 

In Section we list the sources of the observational and
simulated data to be used throughout the paper.
Next, in Section we describe the methods we use to quantify and
characterize the satellite distributions.
In section we present the results. 
In the discussion section we quantify the correlations between the main
plane properties as described by the simulations.
We use this results to quantify the degree of atipicality of the LG
and estimate the volume that has to probed in simulations in order to
find a pair with a satellite distribution as atipical as the LG. 
Finally, we summarize our conclusions in Section .


\section{Data samples}

\subsection{Observational Data}
\label{sec:obs}




\subsection{Data from the Illustris project}
\label{sec:illustris}

We use publicly available data from the Illustris Project 
\citep{2014MNRAS.444.1518V}. 
This suite of cosmological simulations, performed using the quasi-Lagrangian
code AREPO \citep{2010MNRAS.401..791S}, followed the coupled evolution of dark 
matter and gas and includes parametrizations to account for the effects of
gas cooling, photoionization, star formation, stellar feedback, black
hole and super massive black hole feedback. 
The simulation volume is a cubic box of $75$ \Mpch\ on a side.
The cosmological parameters correspond to a $\Lambda$CDM cosmology
consistent with WMAP-9 measurements \citep{2013ApJS..208...19H}. 

We extract halo and galaxy information from the Illustris-1 simulation
which has the highest resolution in the current release of the
Illustris Project.
Illustris-1 has $1820^3$ dark matter particles and $1820^3$ initial gas
volumen elements. 
This corresponds to a dark matter particle mass of
$6.3\times 10^6$\Msun\ and a minimum mass for the baryonic volume
element of $8.0\times 10^7$\Msun. 
The corresponding spatial resolution is $1.4$ kpc for the dark matter
gravitational softening and $0.7$ kpc for the typical size of the
smallest gas cell size. 

We buid a sample of Isolated Pairs that resemble the conditions
in the LG.
To construct this sample we select first all galaxies with  an stellar mass in the range $1\times10^{10}\Msun
<M_{\star}<1.5 \times 10^{11} \Msun$.
Then we consider the following criteria for all galaxies in that set.


\begin{itemize}
\item For each galaxy $A$ we find its closest galaxy $B$, if galaxy $A$ is also
the closest to halo $B$, the two are considered as a pair. 
\item With $d_{AB}$ the distance between the two galaxies and
  $M_{\star,min}$ the lowest stellar mass in the two galaxies, we
  discard pairs that have any other galaxy $C$ with stellar mass
  $M_{\star}>M_{\star, min}$ closer than $3\times d_{AB}$ from any of
  the pair's members. 
\item The distance $d_{AB}$ is greater than $700$ kpc.
\item The relative radial velocity between the two galaxies, including
  the Hubble flow, is $-120\ \kms <v_{AB,r}<0\ \kms$. 
\end{itemize}

We find 27 pairs with these conditions. 
We then select the pairs where in both halos there are at least 15
detected subhalos, thus discarding pairs with halos with the lowest
mass.
We end up with a total of 20 pairs that fulfill these criteria,
Appendix A shows the physical  properties (stellar masses, maximum
circular velocities, radial velocities and separation) in those pairs. 
This corresponds to a pair number density of $2 \times10^{-5}$
pairs Mpc$^{-3}$ 


Although Illustris-1 has stellar particles, we do not use their
properties to select the satelite population because the smallest
galaxies are barely resolved in stellar mass at magnitudes of
$M_V=9$. We prefer using the dark matter information as the smallest
sub-halos are sampled with at least $35$ particles. 
We chose the satellite samples by ranking the subhalos in decreasing
order of its maximum circular velocity and select the first $N_p$
halos in the list. 
The results presented here correspond to $11\leq N_p\leq 15$. 

\subsection{Data from the ELVIS project}
\label{sim:ELVIS}

For a detailed description of the ELVIS project and data we refer the
reader to. 
Here we summarize the elements relevant to our discussion.
The data we use from the ELVIS project comes from the resimulation of
dark matter halo pairs selected in dark matter only cosmological simulations.
The parent cosmological boxes have a cosmology consistent with the
Wilkinson Microwave Anisotropy Probe 7 results.
They used the results from 50 simulation boxes of side lenght $70.4$
Mpc to select pairs with kinematic characteristics similar to the LG.
These selection criteria included the following
\begin{itemize}
\item The virial mass of each host must be in the range 
$1\times
  10^{12} M_{\odot}< M_{vir}<3\times 10^{12}M_{\odot}$ 
\item The total pair mass must be in the range
$2\times
  10^{12} M_{\odot}< M_{vir}<5\times 10^{12}M_{\odot}$ 
\item The center of mass separation is in the range $0.6\leq d\leq1$
  Mpc.
\item The relative radial velocity is negative.
\item No halos more massive than the least massive halo within $2.8$
  Mpc and no halos with $M_{vir}>7\times 10^{13}$ within $7$ Mpc of
  the pairs' center of mass.
\end{itemize} 

This corresponds to a pair number density of $8 \times10^{-6}$ pairs
Mpc$^{-3}$, this is a factor $2.4$ lower than the pair number density
we find in Illustris-1.
There were a total of 146 pairs that met those criteria, but only $12$
were chosen for resimulation. 
Additionally, the selected pairs for resimulation have a relative
tangential velocity less than $75$ \kms. 
In this paper we only use the results from the $12$ resimulated pairs.


\section{Building, Characterizing and Comparing Satellites Spatial Distributions}
\label{sec:SpatialMeasurements}


\subsection{Building Satellite Samples}

We compare the joint satellite distributions in the MW and M31 at fixed
satellite number, $N_s$.
This means that the magnitude cut corresponding to the faintest
satellite included in the sample is different in each case.
We make this choice for two reasons. 
First, to be sure that there is a non-zero number of satellites in the
simulations to make the computations.  
Second, to rule out the influence of satellite numbers in the
statistics. 

We compute the satellite statistics fo 11 up to 15 satellites.
The lowest bound corresponds to the number of classical Milky Way
satellites.
The upper limit corresponds to the maximum number of satellites that
can be resolved in both halos for most of the isolated pairs in Illustris-1.
In simulations we rank the subhalos by their maximum
circular velocity, in observations we rank the satellites by its $M_V$
magnitude. 

We also use two kinds of satellite distributions. 
The first keeps the positions for the satellites fixed as provided in
the observations/simulations; the second randomizes the angular positions of
the satellites around the central galaxy while keeping its radial
distance fixed. The randomization process is done 1000 times for each
galaxy. 



\subsection{Describing Samples with the Inertia Tensor}
We base all our results on the description provided by the inertia
tensor defined by the satellites's positions.  

\begin{equation}
{\bf{\bar{I}}} = \sum_{k=1}^{N_s}[(\bf{r}_i - \bf{r}_0)^2\cdot \bf{1} -
  (\bf{r}_i-\bf{r}_0)\cdot (\bf{r}_i - \bf{r}_0)^{T}],
\end{equation}
%
where $k$ indexes the set of satellites of interest
$\bf{r}_k$ are the satellites' positions, $\bf{r}_{0}$ is the location
of the central galaxy $\bf{1}$ is the unit matrix,  and  
${\bf r}^T$ is the transposed vector $\bf{r}$. 
We use $\bf{r}_0$ as the position of the central galaxy, and not the
satellites' geometrical center, to allow for a fair comparison once
the angular positions of the satellites are randomized around this
point. 

From this tensor we compute its eigenvalues,
$\lambda_1>\lambda_2>\lambda_3$, and corresponding eigenvectors,
$\hat{I}_1$, $\hat{I}_2$, $\hat{I}_3$.
We define the size of the three ellipsoidal axis as
$a=\sqrt{\lambda_1}$, $b=\sqrt{\lambda_2}$ and $c=\sqrt{\lambda_3}$.
We also define $\hat{n}\equiv \hat{I}_1$ as the vector perpendicular to the
planar satellite distribution. 
We also define the width, $w$, of the planar satellite distribution,
$\sigma_p$ as the standard deviation of all satellite distances to the
plane defined by the vector $\hat{n}$. 



To summarize we characterize the satellite distribution by for
quantities obtained from the inertia tensor: 
\begin{itemize}
\item Plane width, $w$.
\item $c/a$ axis ratio.
\item $b/a$ axis ratio.
\end{itemize}


\subsection{Comparing Satellite Samples}

We compare every satellite distribution against its own spherically
randomized distribution.
We keep fixed the radial position of every satellite
with respect to the central galaxy and then randomizing its angular
position. 
We repeat this procedure 1000 times for each satellite distribution
and proceed to measure the quantities mentioned in the last section:
$w$, $c/a$ and $b/a$.
For each quantity we compute an average and standard deviation from
the 1000 random samples. 
This allows us to build a normalized version of all quantities of
interest by substracting the mean and dividing between the standard
deviation of the randomized samples.

This allows us to make a first comparison. 
The observed/simulated
distribution against its randomized version. 
The second intermediate comparison we make are observations against
simulations.
The final comparison is between the normalized quantities, both
observed and simulated.

The comparison between the normalized quantities is the one that
carries the important information about the deviations from
spheriticy. 
We do not want to directly compare how the results observations
deviate from simulations but to compare the deviations from
asphericity in observations and simulations. 


\subsection{Describing joint satellite distributions}

After building the normalized variables with the simulated data we
perform  a Kolmogorov-Smirnov test with the null hypothesis of belongig
to a normal distribution with mean and standard deviation computed
from the mean and standard deviation from the data itself. 
We find that the distributions for $w$, $c/a$ and $b/a$ cannot reject
the null hypothesis. 

We then build a multivariate normal distribution for the two halos and
the three variables consistent with a unidimensional normal
distribution using the vector of mean values and covariance matrix.
This means that our final covariance matrix is $6\times 6$ and takes
into account the correlations internal to a single galaxy and with its
partner. 

This compact description allows us to generate samples consistent with
the simulations used in their construction.
Finally, we use those generated samples to estimate how commoon are
the deviations from sphericity that we measure in the observational
data. 




% referencia posiciones satellites
% http://adsabs.harvard.edu/abs/2013MNRAS.435.1928P

\begin{figure*}
\centering
\includegraphics[width=0.32\textwidth]{scatter_random_ranked_width.pdf}
\includegraphics[width=0.32\textwidth]{scatter_ranked_width.pdf}
\includegraphics[width=0.32\textwidth]{scatter_norm_ranked_width.pdf}
\caption{Plane width characterization in the Local Group and the
  Isolated Pairs. In all panels the horizontal axis corresponds to the
  M31 or the most massive halo in the pair and the vertical axis to
  the MW or the least massive halo in the pair.
The panel on the left shows the plane width in physical units
comparing the results from observations
(stars) against the result of spherically randomizing the satellite
positions while keeping its radial distance (circles). 
The panel in the middle compares the average from the observations
(star) and the average from each one of the Isolated Pairs (circles
with error bars).
The panel on the right has the same information as the middle panel,
only that this time each point has been normalized (median substracted
and normalized by the standar deviation) to the results of its
randomization. 
The main message of this series of plots is that the MW has a
significantly thiner plane both compared to the result of its own
satellite spherical randomization (left panel) and the expectation from
simulations (middle panel). 
This low value is $2\sigma$ away from what is expected in a spherical
distribution. 
In the M31 their satellites are in agreement with the expectations
both from an spherical distribution and the results form the
simulations. 
A second conclusion is that the spherically averaged plane width
MW (seen in the point cloud in the left panel) is smaller than the
average expectation from simulations, while for M31 the spherical
average is consistent with simulations. 
\label{fig:scatter_width}}
\end{figure*}

\begin{figure*}
\centering
\includegraphics[width=0.32\textwidth]{scatter_random_ranked_ca_ratio.pdf}
\includegraphics[width=0.32\textwidth]{scatter_ranked_ca_ratio.pdf}
\includegraphics[width=0.32\textwidth]{scatter_norm_ranked_ca_ratio.pdf}
\caption{Same layout as in Figure \ref{fig:scatter_width}. 
This time for the $c/a$ axis ratio. 
The message holds in this case as for the plane width.
The MW is has a significantly low $c/a$ value compared to the
expectation from a spherical distribution and simulations. 
This low value is also $2\sigma$ away from the expactations for an
spherical distribution.
M31 is consistent both with an spherical distribution and the results
from simulations.
However, in this case the axis ratio in the spherically averaged case
is completeley consistent with the expectation from simulations.
\label{fig:scatter_ca_ratio}}
\end{figure*}


\begin{figure*}
\centering
\includegraphics[width=0.32\textwidth]{scatter_random_ranked_ba_ratio.pdf}
\includegraphics[width=0.32\textwidth]{scatter_ranked_ba_ratio.pdf}
\includegraphics[width=0.32\textwidth]{scatter_norm_ranked_ba_ratio.pdf}
\caption{Same layout as in Figure \ref{fig:scatter_width}. 
This time for the $b/a$ axis ratio. In this case both the MW and M31
are consistent with the results of a spherical distribution and the
simulations. 
\label{fig:scatter_ba_ratio}}
\end{figure*}


\begin{figure*}
\centering
\includegraphics[width=0.45\textwidth]{gaussian_model_illustris_M31.pdf}
\includegraphics[width=0.45\textwidth]{gaussian_model_illustris_MW.pdf}
\caption{Correlations between the multivariate gaussian model built on
  the normalized values for the plane width $w$, $c/a$ ratio and $b/a$ ratio. 
Left/right panel correspond to M31/MW. 
The contour levels in the 2D histograms correspond to the $1\sigma$,
$2\sigma$ and $3\sigma$ contours in two dimensions. 
The dahsed vertical lines in the histograms along the diagonal
correspond to the $1\sigma$ boundaries in one dimension.
The results for the gaussian model are built from $10^6$ point
realizations in the six-dimensional space spaned by the variables of
interest. 
The correlation matrix and the mean values are computed from the
results in the Illustris-1 simulation.
The cross indicates the LG values.
This plot clearly shows how the M31 results are well within the
expectations from simulations while MW has an unusual low value for
the plane width and the $c/a$ axis ratio.
Equivalent results from the ELVIS data are presented in Figure
\ref{fig:correlations_elvis}.
\label{fig:correlations_illustris}}
\end{figure*}


\begin{figure}
\centering
\includegraphics[width=0.47\textwidth]{expected_numbers.pdf}
\caption{Probability distribution for the expected number of pairs
  showing the same degree of atipicality as the Local Group if drawn
  from a sample of 2000 isolated pairs. 
  The distributions correspond to results derived from Illustris-1
  and ELVIS data.
  On average between $0.05\%$ and $0.25\%$ of the isolated pairs should present
  satellite distributions as atypical as the Local Group.
\label{fig:expected_number}}
\end{figure}

\section{Results}
\label{sec:results}


\subsection{Plane Width}

Figure \ref{fig:scatter_width} summarizes the results for the plane
width distributions.
The panel on the left compares the results for the MW and M31
observations against its randomized version. 
The most interesting outcome is that the MW plane width is smaller
than $\approx 98\%$ of the planes computed from the randomized distribution,
while the M31 plane width is consistent with the same distribution. 

\subsection{$c/a$ axis ratio}
Figure \ref{fig:scatter_ca_ratio} shows the results for the minor to
major axis ratio. 
Corresponding in the ELVIS data are in the appendix in Figure
\ref{fig:scatter_elvis}.

The left panel shows the results for the LG compared against its
spherically randomized version.
As expected the $c/a$ ratio in the MW is significantly lower as the
measured values for spherical distributions. 
On the other hand the ratio for M31 is lower than the mean of the
spherical values but still well within its variance.
In contrast to the results for the plane width, this time the
randomized values have symmetrical expectations between the two
galaxies.

The middle panel shows the LG compared against the results from
Illustris. In this case we find a similar trend. The MW is atypical
and M31 is within the variance from the simulation data.
This time, however, there is a single MW-like galaxy out of the total
of 20 that shows an $c/a$ as small as the MW.
The only assymetry evident between the simulated halos is in the
dispersion, the $c/a$ distribution for the massive halos seems to be
wider than it is for the less massive partner. 

The right panel shows the normalized results. 
This highlights the two results mentioned above.
First, the MW shows a low $c/a$ ration close to two standard deviations away
from the mean value of the spherical distribution; this contrasts the
results for M31 which are close to $1$ standard deviation away.
Second, the less massive halos in the Illustris simulation present a
smaller dispersion in their deviations from sphericity than the
its masssive partner.
The MW axis ratio smaller than $\approx 98\%$ of the randomized
satellite distributions.   and its atipicallity is only comparable to
one halo in the Illustris  simulation (none in the ELVIS simulation).


\subsection{$b/a$ axis ratio}

Figure \ref{fig:scatter_ba_ratio} shows the results for the minor to
major axis ratio. 
Corresponding in the ELVIS data are in the appendix in Figure
\ref{fig:scatter_elvis}.
In all cases of comparison (against randomized distribution,
comparasion against simulations and degree of deviation from
sphericity) the results for both the MW and M31 do not seem atypical
or deviated from the expectations for an spherical dsitribution.



\subsection{Multivariate Distributions}


\section{Discussion}

\begin{table*}
  \centering
  \renewcommand{\arraystretch}{1.2}
  \begin{tabular}{|p{2.5cm}|c|c|c|c|c|c|c|c|}
    \hline
    \multirow{2}{4.0cm}{} & \multicolumn{2}{c|}{\textbf{Observations}} & \multicolumn{2}{c|}{\textbf{Randomized Obs.}} & \multicolumn{2}{c|}{\textbf{Illustris-1}} & \multicolumn{2}{c|}{\textbf{ELVIS}}\\
    % \hline
    % \textbf{Inactive Modes} & \textbf{Description}\\
    \cline{2-9}
    & \textbf{M31} & \textbf{MW} & \textbf{M31} & \textbf{MW} & \textbf{M31} & \textbf{MW}& \textbf{M31} & \textbf{MW}\\
    %\hhline{~--}
    \hline
    Plane width (kpc) & $59\pm 3$  & $22\pm 2$  & $64\pm 12$   & $45\pm 8$     & $70\pm 4$ & $67\pm 2$ & $70\pm 2$& $68\pm 4$ \\\hline
    $c/a$ ratio & $0.45\pm 0.04$ & $0.28\pm 0.03$ & $0.55\pm0.10$ & $0.53\pm 0.10$ & $0.52\pm 0.01$ & $0.53\pm 0.01$ & $0.54\pm 0.01$& $0.49\pm 0.02$ \\ \hline
    $b/a$ ratio & $0.82\pm 0.06$ & $0.78\pm 0.02$ & $0.82\pm0.07$ & $0.81\pm 0.08$ & $0.80\pm 0.01$ & $0.80\pm 0.02$ & $0.80\pm0.01$& $0.81\pm 0.01$\\ \hline
  \end{tabular}
  \caption{Mode Transition Times}
\end{table*}





\subsection{Outlier characterization with a Multivariate Gaussian Model}


Prospects for observational measurement: DESI.

\section{Conclusions}

In this paper we develop and demostrate a method to quantify the
asphericity of the satellite distribution in the Local Group.
The method uses as a basic condition the spherical randomization of
satellite positions (observed or simulated). 
The scalars describing the satellite distribution in each galaxy are
then normalized to the mean value and standard deviation in the
randomized samples. 
In observations we limit our analysis to the 11 brightest satellites 
in the Milky Way and Andromeda. 
For numerical data we use the Illustris-1 and the ELVIS simulations. 

We find that the deviation from aesphericity in the LG is only
expected in $1\pm3$ pairs out of a sample of $2000$ isolated pairs. 
This places the LG as a $3\sigma$ outlier. 
The weight to explain this atypical result is not distributed equally
between the MW and M31. While M31 presents a fully typical asphericity
in the expectations from LCDM, the MW shows aspherical deviations in
plane width and the major-to-minor axis ratio highly atypical in the
framework of LCDM. 
We estimate that with the M31 $XX$ out of $2000$ pairs show normalized
characteristics larget r equal than M31, while this number drops to 
$XX$ for the MW.

The method we present to quantify how atypical is the LG is robust to
the numerical simulations used to define the LCDM expectations. 
Quantifying the degree of asphericity highlights the atypicality of
the Milky Way stressing the stark difference with M31 and its more
typical nature. 
This atypical distribution also adds up to the presence of two bright
satellite galaxies, which is also uncommon both in simulations and
observations. 

The extension of the framework we present in this paper to describe
outliers found as a second order deviation (i.e. fiding a \emph{subset} of
satellites in M31 that are in a plane or \emph{adding} velocity
information) could also be possible along the line of work presented
by, that avoid finding in observations the exact observed distribution
and instead focuse in defining atypicallity. 


The quantitative study we present allows us to estimate  the volumes 
that cosmological simulations have to probe in order to study similar
LG configurations.
Such atypical satellite distribution should be seen as an opportunity
to constraint in high detail the initial conditions and environment
that allowed such pattern to emerge. 




\bibliographystyle{mnras}
\bibliography{Dwarfs}

%% Alignments between galaxies, satellite systems and haloes
%% https://arxiv.org/pdf/1605.01728.pdf

%M31 mass
%% https://arxiv.org/abs/1410.0017

%MW mass
%https://arxiv.org/abs/1407.1078
\newpage
\appendix

\section{Physical Characteristics of the Isolated Pairs samples}

\section{Results from ELVIS}
\begin{figure*}
\centering
\includegraphics[width=0.30\textwidth]{scatter_ranked_elvis_width.pdf}
\includegraphics[width=0.30\textwidth]{scatter_ranked_elvis_ca_ratio.pdf}
\includegraphics[width=0.30\textwidth]{scatter_ranked_elvis_ba_ratio.pdf}
\includegraphics[width=0.30\textwidth]{scatter_norm_ranked_elvis_width.pdf}
\includegraphics[width=0.30\textwidth]{scatter_norm_ranked_elvis_ca_ratio.pdf}
\includegraphics[width=0.30\textwidth]{scatter_norm_ranked_elvis_ba_ratio.pdf}
\caption{ELVIs results for the quantities presented for the Illustris-1
  simulation in Figures  \ref{fig:scatter_width},
  \ref{fig:scatter_ca_ratio}, \ref{fig:scatter_ba_ratio}.
Upper row corresponds to the raw values from observations and
simulated pairs, while the second row normalizes the same values to
the mean and standard deviation on its spherically randomized
counterparts. 
\label{fig:scatter_elvis}}
\end{figure*}

\begin{figure*}
\centering
\includegraphics[width=0.45\textwidth]{gaussian_model_elvis_M31.pdf}
\includegraphics[width=0.45\textwidth]{gaussian_model_elvis_MW.pdf}
\caption{
Same layout as Figure \ref{fig:correlations_illustris}, this time
computed from the ELVIS data.
\label{fig:correlations_elvis}}
\end{figure*}




Illustris
\begin{verbatim}
[[ 0.8344  0.7849  0.4186 -0.1155  0.0340 -0.1614]
 [ 0.7849  1.1698 -0.1568 -0.0793  0.0362 -0.1213]
 [ 0.4186 -0.1568  0.9606 -0.1151 -0.0150 -0.1098]
 [-0.1155 -0.0793 -0.1151  0.7873  0.6339  0.4006]
 [ 0.0340  0.0362 -0.0150  0.6339  0.6932  0.0652]
 [-0.1614 -0.1213 -0.1098  0.4006  0.0652  0.6100]]
\end{verbatim}

\begin{verbatim}
[-0.1852 -0.4152    -0.2002 -0.1786 -0.4072 -0.2394]
\end{verbatim}

ELVIS
\begin{verbatim}
[[ 1.6051  1.2207  0.7062  0.0833  0.1063  0.0812]
 [ 1.2207  1.0872  0.3584  0.0298  0.1085 -0.0324]
 [ 0.7062  0.3584  0.6363  0.0904  0.0194  0.1601]
 [ 0.0833  0.0298  0.0904  0.4534  0.6138 -0.1159]
 [ 0.1063  0.1085  0.0194  0.6138  1.2120 -0.6372]
 [ 0.0812 -0.0324  0.1601 -0.1159 -0.6372  0.6475]]
\end{verbatim}

\begin{verbatim}
[-0.1803 -0.2829 -0.2205 -0.3292 -0.7484 -0.0863]
\end{verbatim}
\end{document}

