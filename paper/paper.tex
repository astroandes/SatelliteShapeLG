\documentclass[a4paper,fleqn,usenatbib]{mnras}

%=========================================================================
\usepackage{amsmath} 
\usepackage{amssymb} 
\usepackage{multirow}

\usepackage{graphicx}
\usepackage{grffile}
\usepackage[dvips]{epsfig}
\usepackage{epsfig}  
\usepackage{color}
\usepackage{caption}
\usepackage{hyperref}
\usepackage{bm}
%Non reposionated tables

%=========================================================================
%		INTERNAL MACROS
%=========================================================================
\def\be{\begin{equation}}
\def\ee{\end{equation}}
\def\ba{\begin{eqnarray}}
\def\ea{\end{eqnarray}}

% To highlight comments 
\definecolor{red}{rgb}{1,0.0,0.0}
\newcommand{\red}{\color{red}}
\definecolor{darkgreen}{rgb}{0.0,0.5,0.0}
\newcommand{\SRK}[1]{\textcolor{darkgreen}{\bf SRK: \textit{#1}}}
\newcommand{\SRKED}[1]{\textcolor{darkgreen}{\bf #1}}
\newcommand{\before}[1]{\textcolor{red}{ #1}}
\newcommand{\after}[1]{\textcolor{darkgreen}{ #1}}
\newcommand{\hs}{{\hspace{1mm}}}  
\newcommand{\tol}{Tololo 1214-277}
\newcommand{\rank}{\texttt{Ranked}}
\newcommand{\boot}{\texttt{Bootstrapped}}
\newcommand{\rand}{\texttt{Random}}
\newcommand{\HI}{{\text{H\MakeUppercase{\romannumeral 1}}} }
\newcommand{\HII}{{\text{H\MakeUppercase{\romannumeral 2}}} }
\newcommand{\lya}{\ifmmode{{\rm Ly}\alpha}\else Ly$\alpha$\ \fi}
\newcommand{\cm}{\ifmmode{{\rm cm}}\else cm\fi}
\newcommand{\ccm}{\,\mathrm{cm}^{-3}}
\newcommand{\ergps}{\,{\rm erg}\,{\rm s}\ifmmode{}^{-1}\else ${}^{-1}$\fi}
\newcommand{\Mpch}{\,{\rm Mpc}\,\ifmmode h^{-1}\else $h^{-1}$\fi}
\newcommand{\dd}{\mathrm{d}}
\newcommand{\vek}[1]{\bm{#1}}
\newcommand{\hb}{H$\beta$}
\newcommand{\ha}{H$\alpha$}
\newcommand{\oiii}{[OIII]}
\newcommand{\oii}{[OII]}
\newcommand{\nii}{[NII]}
\newcommand{\esca}{erg cm$^{-2}$ s$^{-1}$ \AA$^{-1}$}
\newcommand{\esc}{erg cm$^{-2}$ s$^{-1}$}
\newcommand{\es}{erg s$^{-1}$}
\newcommand{\esa}{erg s$^{-1}$}
\newcommand{\kms}{\ifmmode\mathrm{km\ s}^{-1}\else km s$^{-1}$\fi}
\newcommand{\hMsun}{{\ifmmode{h^{-1}{\rm{M_{\odot}}}}\else{$h^{-1}{\rm{M_{\odot}}}$}\fi}}
\newcommand{\Msun}{{\ifmmode{{\rm{M_{\odot}}}}\else{${\rm{M_{\odot}}}$}\fi}}

\newcommand{\jefr}[1]{\textcolor{darkgreen}{\bf JEFR: \textit{#1}}}

\begin{document}

%=========================================================================
%		FRONT MATTER
%=========================================================================
\title[LG satellites distribution asphericity]{We are not the 99 percent: quantifying
  asphericity in the distribution of Local Group satellites}
\author[J.E. Forero-Romero \& V. Arias]
{Jaime E. Forero-Romero $^{1}$ \thanks{je.forero@uniandes.edu.co},
Ver\'onica Arias$^1$\\
%%
$^1$ Departamento de F\'isica, Universidad de los Andes, Cra. 1
  No. 18A-10 Edificio Ip, CP 111711, Bogot\'a, Colombia \\
}

\maketitle

\begin{abstract}
We use simulations to build an explicit probability distribution for
the asphericity in the satellite distribution around Local Group (LG)
type galaxies in the Lambda Cold Dark Matter (LCDM) paradigm. 
We use this distribution to estimate the  atypicality
of the satellite distributions in the LG even when the underlying
simulations do not have enough systems fully resembling the LG.
We demonstrate the method using three different simulations
(Illustris-1,  Illustris-1-Dark and ELVIS) and a number of satellites
ranging from 11 to 15.
Detailed results differ greatly among the simulations suggesting a
strong influence of the typical DM halo mass, the number of satellites
and the simulated baryonic effects.   
However, there are three common trends.
First, at most $2\%$ of the pairs are expected to have satellite
distributions with the same asphericity as the LG; second,
at most $80\%$ of the pairs have a halo with a satellite
distribution as aspherical as in M31; and third, at most $4\%$ of the
pairs have a halo with satellite distribution as planar as in the MW. 
These quantitative results place the LG at the level of a $3\sigma$
outlier in the LCDM paradigm. 
We suggest that understanding the reasons for this atypicality
requires quantifying the asphericity probability distribution as a
function of halo mass and large scale environment.
The approach presented here can facilitate that kind of study and other
comparisons between different numerical setups and choices to study
satellites around LG pairs in simulations.  
\end{abstract}

\begin{keywords}Cosmology: Dark Matter --- Galaxies:Local Group ---
  Methods: numerical  
\end{keywords}

\section{Introduction}

The spatial distribution of satellite galaxies around our Milky Way
(MW) and the M31 galaxy is becoming a stringent test for structure
formation theories in an explicit cosmological context. 
This started with the suggested existence of a Magellanic Plane, a flattened
structure of satellite galaxies and globular clusters around the MW,
by \cite{1976RGOB..182..241K} and \cite{1976MNRAS.174..695L}.  
Fourty years later \cite{2005A&A...431..517K} quantified that the
highly planar distribution of the 11 classical MW satellites has less
than a $0.5\%$ chance to happen by chance if the
parent distribution is spherically symmetric, interpreting this as a
challenge to the Lambda Cold Dark Matter (LCDM) paradigm.
The same year it was recognized, using numerical simulations, that
this comparison was unfair given that Dark Matter halos in LCDM
are expected to be triaxial and not spherical.
Nevertheless, some estimates of the chances to find simulated satellites as
planar as in the MW 
were as low as $2\%$ \citep{2005ApJ...629..219Z} while others expected
planar satellite configurations in every DM halo
\citep{2005MNRAS.363..146L}.

Later, \cite{2007MNRAS.374.1125M} used simulations to confirm the low
chances ($<0.5\%$) found in \cite{2005A&A...431..517K}.
This result was challenged by \cite{2009MNRAS.399..550L} who continued
to report high chances to find a planar configuration;
two more recent numerical experiments with high resolution simulations
(\cite{2013MNRAS.429..725S} and \cite{2016MNRAS.457.1931S}) reported
contradicting results (low and high chances to find the observational
result, respectively) but precise figures were not quoted in these three reports. 
Meanwhile, \cite{2013MNRAS.429.1502W} and \cite{2014ApJ...789L..24P} published
the results of new tests using high resolution simulations and cosmological
volumes arriving at a chance of $6\%$ and $0.77\%$, respectively, to
have a satellite distribution as triaxial as the observations. 


In the case of M31, all studies agree on the point that the spatial
distribution of the 15 to 27 brightest satellites  are consistent with
spherically symmetric distributions and easy to reproduce in LCDM simulations
\citep{2006AJ....131.1405K,2007MNRAS.374.1125M, 2013ApJ...766..120C}.
In a very different observational statement, there is  subset of $15$
satellites out of the  total poulation of $27$ satellites
\citep{2013ApJ...766..120C, 2013Natur.493...62I} that shows a very
planar distribution. 
However, in this paper we do not address these \emph{plane finding}
algorithms and focus instead on the characterization of the satellite
distribution ranked by decreasing luminosity, maximum circular
velocity or stellar mass.  


Table \ref{table:M31} and Table \ref{table:MW} summarize some more
details on the results we have just mentioned \footnote{
Those tables also include the results from this paper using the
methodology we describe in the upcoming sections.}.

\begin{table*}
\centering
\begin{tabular}{|p{4.0cm}|p{4.5cm}| p{5.5cm}| c|}\hline
Reference & Target Measurement & Parent Simulation & Probability ($\%$)\\\hline
\text{\cite{2006AJ....131.1405K}} & RMS width in 15 brightest
satellites & Monte Carlo satellite distributions with a power law
radial distribution & $87-99$\\
\text{\cite{2007MNRAS.374.1125M}} & $c/a$ ratio and RMS width in 16 brightest satellites & Monte Carlo from a spherical power law radial
distribution & $17$\\
\text{\cite{2013ApJ...766..120C}}& RMS width in the 27 brightest
satellites & Monte Carlo randomized satellite distribution & High\\
This Work & $c/a$ ratio, $b/a$ ratio and RMS width in 11-15 brightest
satellites & 27 halos from selected pairs in a cosmological N-body Dark Matter only simulation ($\sim
10^{6}$\Msun particle mass resolution)& $<80$ \\
This Work & $c/a$ ratio, $b/a$ ratio and RMS width in 11-15 brightest
satellites & 24 halos from selected pairs in a cosmological N-body hydro simulation ($\sim
10^{6}$\Msun particle mass resolution)& $<71$ \\
This Work & $c/a$ ratio, $b/a$ ratio and RMS width in 11-15 brightest
satellites & 12 high resolution DM only N-body simulations ($\sim
10^{5}$\Msun particle mass resolution) of halo pairs & $<57$ \\
\hline
\end{tabular}
\caption{Probability to find  the triaxiality and/or root mean squared
  (RMS) height of M31 satellites in LCDM simulations. 
\label{table:M31}}
\end{table*}


\begin{table*}
\centering
\begin{tabular}{|p{4.0cm}|p{4.5cm}| p{5.5cm}| c|}\hline
Reference & Target Measurement & Parent Simulation & Probability ($\%$)\\\hline
\text{\cite{2005A&A...431..517K}} & RMS width in 11 classical
satellites & Monte Carlo from a spherical power law
radial distribution. & $<0.5$ \\
\text{\cite{2005MNRAS.363..146L}} & $c/a$ ratio of 11 classical
satellites & 6 high resolution DM only N-body simulations ($\sim10^5$
\Msun\ particle mass resolution). & High\\ 
\text{\cite{2005ApJ...629..219Z}} & $c/a$ ratio and disk height in 
11 classical satellites & 3 high resolution DM only N-body
simulations. ($\sim10^6$ \Msun\ particle mass resolution) & 2 \\
\text{\cite{2007MNRAS.374.1125M}} & $c/a$ ratio and RMS width in 11-13 brightest satellites & Monte Carlo from a halo triaxiality distribution from LCDM
simulations & $<0.5$\\
\text{\cite{2009MNRAS.399..550L}}& $c/a$ ratio in 11 classical satellites & 436 halos from a
cosmological N-body simulation ($1.3\times 10^{8}$\Msun\ particle mass)
&  High \\
\text{\cite{2013MNRAS.429..725S}}& $c/a$ ratio in 12 brightest
satellites & 6 High resolution DM only N-body simulations ($\sim
10^3$\Msun particle mass resolution) of individual halos & Low \\
\text{\cite{2013MNRAS.429.1502W}}& $c/a$ ratio or RMS width in 11 brightest
satellites & 1686 halos from a cosmological DM only N-body simulation
($\sim 10^6$\Msun particle mass resolution) & 6 or 13 \\
\text{\cite{2014ApJ...789L..24P}}& $c/a$ and $b/a$ ratio in 11
classical satellites & 48 high resoltion DM only N-body simulations
($\sim 10^{5}$\Msun particle mass resolution) of both halo pairs and
isolated halos & 0.77\\
\text{\cite{2016MNRAS.457.1931S}}& $c/a$ ratio in 11 classical satellites & 12
high resolution Hydro simulation of halo pairs ($\sim 10^{4}$\Msun particle
mass resolution) & High\\
This Work & $c/a$ ratio, $b/a$ ratio and RMS width in 11-15 brightest
satellites & 27 halos from selected pairs in a cosmological N-body
Dark Matter only simulation ($\sim 10^{6}$\Msun particle mass
resolution)& $<4$\\
This Work & $c/a$ ratio, $b/a$ ratio and RMS width in 11-15 brightest
satellites & 24 halos from selected pairs in acosmological N-body
hydro simulation ($\sim 10^{6}$\Msun particle mass resolution)& $<1.5$ \\
This Work & $c/a$ ratio, $b/a$ ratio and RMS width in 11-15 brightest
satellites & 12 high resolution DM only N-body simulations ($\sim
10^{5}$\Msun particle mass resolution) of halo pairs & $<1.6$\\
\hline
\end{tabular}
\caption{Same as Table \ref{table:M31} for the MW satellites.
\label{table:MW}}
\end{table*}


Some of the difficulty in trying to reconcile and understand the
seemingly conflicting or inconclusive results on the MW has its origin on
the frequentist fashion generally used to compute probabilities.
Usually, this process starts by building a high level parent sample in the
simulation and then counting how many elements has the subset
meeting some criteria. 
This has two inconveniences.
The first is that the probability estimate is made against whatever
turns out to be typical in each simulation. 
A fair comparison across simulations would require first
characterizing all the simulations at the high level parent samples,
something that is difficult to do in practice. 
The second inconvenient is that the systems that fully resemble
the LG (i.e. its stellar mass content, morphology or kinematics) have
a low cosmological number density, this means that for the current
cosmological volumes in simulations the high level parent sample has a
small size, making it hard to derive robust probabilities by counting.   
In this paper we present and demostrate a method to overcome these two
limitations.

We control the first effect by setting as a direct point of reference
a spherical satellite distribution and not the simulations themselves.
The spherical satellite distribution is built from the data itself
(observational or simulated) by randomizing the angular position
of each satellite around the central galaxy and keeping its radial
distance fixed \citep{2017AN....338..854P}. 
We characterize the satellites in terms of the scalars describing its
deviation from the spherical distribution. 
We then build an explicit analytic probability distribution for the
asphericity; this solves the problem of having a common reference
point to compare simulations.  
We use a simulation to estimate the parameters in this distribution to
later use it as a parent sample to generate any desired number of
samples that are by construction statistically compatible with the
simulation; thus overcoming the problem of having a small number of
systems in the parent simulations. 

To summarize, we use asphericity to characterize on equal footing
simulations and observations.  
Then, we build an explicit probability distribution for the asphericity and
use simulations to estimate its free parameters.
Finally, we use these distributions to generate large numbers of
samples and directly estimate the number of systems meeting a desired
set of criteria.

The rest of the paper describes in detail our implementation and
results. It is structured as follows. 
In Section \ref{sec:DataSamples} we list the sources of the observational and
simulated data to be used throughout the paper.
In Section \ref{sec:SpatialMeasurements} we describe the methods we
use to build halo pairs and  
characterize their satellite distributions.
In Section \ref{sec:results} we present our results to finally
conclude in Section \ref{sec:conclusions}. 



\section{Data samples}\label{sec:DataSamples}

\subsection{Observational Data}
\label{sec:obs}

The base for our analysis is the catalog compiled by
\cite{2014yCat..74351928P} which reports information on all 
galaxies within 3 Mpc around the Sun to that date. 
Detailed description of the compiled catalog can be found in
\cite{2013MNRAS.435.1928P}, here we summarize the relevant features
for the current study.
The information in the catalogue is based on the catalogue compiled by
\cite{2012AJ....144....4M}.
The distance estimates are based on resolved stellar populations. 
We use three dimensional positions in a cartesian coordinate system
as computed by \cite{2013MNRAS.435.1928P}.
In this coordinate system the $z$-axis points towards the Galactic north pole, the
$x$-axis points in the direction from the Sun to the Galactic center,
and the $y$-axis points in the direction of the Galactic rotation.


For both the M31 and MW we only use the 11 to 15 brightest satellites (using
$M_V$ magnitudes) within a distance of $300$kpc from its central galaxy.
The satellites included for the MW analysis are: 
LMC, SMC, Canis Major, Sagittarius dSph, Fornax, Leo I, Sculptor,
Leo II, Sextans I, Carina, Ursa Minor, Draco, Canes Venatici (I),
Hercules and Bootes II.
The satellites included for the M31 analysis are: Triangulum, NGC205,
M32, IC10, NGC185, NGC147, Andromeda VII, Andromeda II, Andromeda
XXXII, Andromeda XXXI, Andromeda I, Andromeda VI, Andromeda XXIII, LGS 3, 
 and Andromeda III.



\subsection{Simulated Data}
 
We use data from three different simulations Illustris-1,
Illustris-1-Dark and ELVIS. 
In what follows we describe the relevant detailed of those
simulations, how the halo pairs resembling the LG are selected and how
the satellite samples for each halo is built.
Figure \ref{fig:physical_pairs} summarizes the physical properties (maximum
circular velocities, radial velocities and separation) of the halo
pairs to be used.

\subsubsection{Illustris-1 and Illustris-1-Dark}
\label{sec:illustris}

We use publicly available data from the Illustris Project 
\citep{2014MNRAS.444.1518V}. 
This suite of cosmological simulations were performed using the quasi-Lagrangian
code AREPO \citep{2010MNRAS.401..791S}.
They followed the coupled evolution of dark  matter and gas and
includes parametrizations to account for the effects of 
gas cooling, photoionization, star formation, stellar feedback, black
hole and super massive black hole feedback. 
The simulation volume is a cubic box with a $75$ \Mpch\ side.
The cosmological parameters correspond to a $\Lambda$CDM cosmology
consistent with WMAP-9 measurements \citep{2013ApJS..208...19H}. 

We extract halo and galaxy information from the Illustris-1 and
Illustris-1-Dark simulations. 
The former includes hydrodynamics and star formation prescriptions while the latter only
includes dark matter physics. 
These simulations have the highest resolution in the current release of the
Illustris Project.
Illustris-1 has $1820^3$ dark matter particles and $1820^3$ initial gas
volumen elements, while Illustris-1-Dark has $1820^3$ dark matter particles.
This corresponds to a dark matter particle mass of
$6.3\times 10^6$\Msun\ and a minimum mass for the baryonic volume
element of $1.2\times 10^6$\Msun\ for Illustris-1 and a dark matter
particle mass of $7.6\times 10^6$\Msun\ for Illustris-1-Dark.
In both simulations the dark matter gravitational softening is $1.4$
kpc.

We buid a sample of pairs that resemble the conditions in the LG as follows.
First, we select from Illustris-1 all the galaxies with a stellar mass
in the range $1\times10^{10}\Msun <M_{\star}<1.5 \times 10^{11} \Msun$.
Then we select the pairs with the following conditions.

\begin{itemize}
\item For each galaxy $A$ we find its closest galaxy $B$, if galaxy $A$ is also
the closest to $B$, the two are considered as a pair. 
\item With $d_{AB}$ the distance between the two galaxies and
  $M_{\star,min}$ the lowest stellar mass in the two galaxies, we
  discard pairs that have any other galaxy $C$ with stellar mass
  $M_{\star}>M_{\star, min}$ closer than $3\times d_{AB}$ from any of
  the pair's members. 
\item The distance $d_{AB}$ is greater than $700$ kpc.
\item The relative radial velocity between the two galaxies, including
  the Hubble flow, is $-120\ \kms <v_{AB,r}<0\ \kms$. 
\end{itemize}

We find 27 pairs with these conditions. 
In Illustris-1-Dark we use the center of mass position of the 27 pairs
in Illustris-1 to find the matching halo pairs.
After discarding the pairs with less than 15 detected subhalos in one
of the halos we end up with a total of 24 pairs in Illustris-1-Dark. 
This corresponds to a pair number density of $\sim 2 \times10^{-5}$
pairs Mpc$^{-3}$. 
Figure \ref{fig:physical_pairs} summarizes the physical properties (maximum
circular velocities, radial velocities and separation) of the halo
pairs to be used.

Although Illustris-1 has stellar particles, we do not use their
properties to select the satelite population for the results reported
here. 
The smallest galaxies are barely resolved in stellar mass at
magnitudes of $M_V=-9$, close to the limit of the 11 ``classical'' MW
satellites, we use instead the dark matter information. 
However, we tested using $M_V$ in Illustris-1 as a selection criteria
and the results we report here are not strongly influenced by that
choice. 

We decide to use both in Illustris-1 and Illustris-1-Dark the same
satellite selection rules.
Namely, selecting the satellites by ranking the subhalos in decreasing
order of its current maximum circular velocity and select the first
$N_s$ halos in the list. 
These selection criteria are also convenient.
They allow us to direct our attention at baryonic physics as the driver
behind the differences in the results between the two simulations. 
Under these conditions the smallest sub-halos are sampled with at
least $40$ particles. This approximation works well for the $\approx 10$ most
massive subhalos in MW type galaxies; to a good approximation these
are the systems with the highest circular velocity at infall
\citep{2011MNRAS.415L..40B}, which is the physical quantity expected
to best correlate with luminosity
\citep{2004ApJ...609...35K,2006ApJ...647..201C,2010MNRAS.404.1111G}. 





\subsection{Data from the ELVIS project}
\label{sim:ELVIS}

We use data from the public release of the Exploring the Local
Universe In Simulations (ELVIS) project.
For a detailed description of that project and its data we refer the
reader to \cite{2014MNRAS.438.2578G}. 
Here we summarize the elements relevant to our discussion.

ELVIS data comes from resimulations of dark matter halo pairs selected
in dark matter only cosmological simulations. 
The parent cosmological boxes have a cosmology consistent with the
Wilkinson Microwave Anisotropy Probe 7 results.

The ELVIS project used the results from $50$ simulation boxes of side
lenght $70.4$ Mpc to select pairs with kinematic characteristics
similar to the LG. 
These selection criteria included the following
\begin{itemize}
\item The virial mass of each host must be in the range 
$1\times
  10^{12} M_{\odot}< M_{vir}<3\times 10^{12}M_{\odot}$ 
\item The total pair mass must be in the range
$2\times
  10^{12} M_{\odot}< M_{vir}<5\times 10^{12}M_{\odot}$ 
\item The center of mass separation is in the range $0.6\leq d\leq1$
  Mpc.
\item The relative radial velocity is negative.
\item No halos more massive than the least massive halo within $2.8$
  Mpc and no halos with $M_{vir}>7\times 10^{13}$ within $7$ Mpc of
  the pairs' center of mass.
\end{itemize} 

This corresponds to a number density of $\sim 8 \times10^{-6}$
pairs/Mpc$^{3}$, which is a factor $\sim 2.5$ lower than the pair
number density we find in the Illustris-1 data.
There were a total of 146 pairs that met those criteria, but only $12$
were chosen for resimulation. 
Additionally, the selected pairs for resimulation have a relative
tangential velocity less than $75 $\kms. 
The dark matter particle resolution in these resimulations is
$1.9\times 10^5$, an order of magnitude times better than Illustris-1.
In this paper we only use the results from these $12$ resimulated pairs.
To be consistent with the satellite selection in Illustris-1 and
Illustris-1-Dark we also select satellites by ranking them by their
maximum circular velocity.


\section{Building, Characterizing and Comparing Satellites Spatial Distributions}
\label{sec:SpatialMeasurements}


\subsection{Building Satellite Samples}

We compare the joint satellite distributions in the MW and M31 at fixed
satellite number, $N_p$.
This means that the magnitude cut corresponding to the faintest
satellite included in the sample is different in each case.
We make this choice to rule out the influence of satellite numbers
in the statistics. 

We compute the satellite statistics for 11 up to 15 satellites.
The lower limit corresponds to the number of classical MW satellites,
while the upper limit corresponds to the minimum number of M31
satellites usually included in M31 studies.
In simulations we rank the subhalos by their maximum circular
velocity, in observations we rank the satellites by its $M_V$
magnitude.  


\begin{table*}
  \centering
  \renewcommand{\arraystretch}{1.2}
  \begin{tabular}{|p{2.5cm}|c|c|c|c|c|c|}
    \hline
    \multirow{2}{4.0cm}{} & \multicolumn{2}{c|}{\textbf{Observations}}
    & \multicolumn{2}{c|}{\textbf{Randomized Obs.}} &
    \multicolumn{2}{c|}{\textbf{Normalized Units}}\\
    % \hline
    % \textbf{Inactive Modes} & \textbf{Description}\\
    \cline{2-7}
    & \textbf{M31} & \textbf{MW} & \textbf{M31} & \textbf{MW} & \textbf{M31} & \textbf{MW} \\
    %\hhline{~--}
    \hline
lane width (kpc) & $63.64$ & $19.34$ & $56.74\pm10.96$ & $37.56\pm7.90$ & $0.63$ & $-2.31$\\\hline
$c/a$ ratio & $0.42$ & $0.18$ & $0.53\pm0.11$ & $0.46\pm0.11$ & $-1.00$ & $-2.46$\\\hline
$b/a$ ratio & $0.96$ & $0.87$ & $0.81\pm0.08$ & $0.80\pm0.09$ & $1.69$ & $0.80$\\\hline
  \end{tabular}
  \caption{Results from observations with $N_s=11$ for the plane width, $c/a$
    ratio and $b/a$ ratio. 
    The first column refers to the results in physical units, the
    second column uses the spherically  randomized version of the 
    observational data and the third column corresponds to the
    observational data re-centered and normalized using the mean and standard
    deviation from the randomized data. Figure \ref{fig:normalized_n}
    shows all the normalized data results for $11\leq N_s\leq
    15$\label{table:observations}.}  
\end{table*}


\begin{figure*}
\centering
\includegraphics[width=0.30\textwidth]{normalized_width_n_dependence.pdf}
\includegraphics[width=0.30\textwidth]{normalized_ca_ratio_n_dependence.pdf}
\includegraphics[width=0.30\textwidth]{normalized_ba_ratio_n_dependence.pdf}
\caption{Normalized asphericity scalars from observations as a function
  of satellite number. Left: plane width; center: $c/a$ ratio;
  right: $b/a$ ratio. 
  The plane width and $c/a$ ratio for the MW is
  consistently found beyond two standard
  deviations.
  M31 shows the opposite trend and is always found
  \emph{within} two standard deviations. \label{fig:normalized_n}} 
\end{figure*}



\begin{table*}
  \centering
  \renewcommand{\arraystretch}{1.2}
  \begin{tabular}{|p{2.5cm}|c|c|c|c|c|c|}
    \hline
    \multirow{2}{4.0cm}{} & \multicolumn{2}{c|}{\textbf{Illustris-1-Dark}} & \multicolumn{2}{c|}{\textbf{Illustris-1}} & \multicolumn{2}{c|}{\textbf{ELVIS}}\\
    % \hline
    % \textbf{Inactive Modes} & \textbf{Description}\\
    \cline{2-7}
    & \textbf{M31} & \textbf{MW} & \textbf{M31} & \textbf{MW}& \textbf{M31} & \textbf{MW}\\
    %\hhline{~--}
    \hline
Plane width (kpc) & $56.28\pm 18.30$ & $58.30 \pm 13.51$  & $66.66\pm 15.75$ & $64.81\pm 15.34$ & $68.74\pm 16.82$ & $63.70\pm 13.62$\\\hline
$c/a$ ratio & $0.48\pm 0.15$ & $0.47 \pm 0.11$  & $0.52\pm 0.13$ & $0.50\pm 0.10$ & $0.54\pm 0.15$ & $0.47\pm 0.13$\\\hline
$b/a$ ratio & $0.80\pm 0.10$ & $0.80 \pm 0.08$  & $0.82\pm 0.09$ & $0.82\pm 0.07$ & $0.81\pm 0.07$ & $0.80\pm 0.09$\\\hline
  \end{tabular}
  \caption{Results from simulations for $N_s=11$. Mean values and standard deviations for the different
    quantities describing the satellite distributions: plane width, $c/a$
    ratio and $b/a$ ratio. 
    The first column summarizes the results from the 27 pairs in
    Illustris-1-Dark, the second column from the 24 pairs in
    Illustris-1 and the third column from the 12 pairs in the
    ELVIS project.\label{table:simulations}}
\end{table*}



\subsection{Describing Samples with the Inertia Tensor}

We describe the satellites with the inertia
tensor defined by the satellites' positions.  

\begin{equation}
{\bf{\bar{I}}} = \sum_{i=1}^{N_p}[(\bf{r}_i - \bf{r}_0)^2\cdot \bf{1} -
  (\bf{r}_i-\bf{r}_0)\cdot (\bf{r}_i - \bf{r}_0)^{T}],
\label{eq:intensor}
\end{equation}
%
where $k$ indexes the set of satellites of interest
$\bf{r}_k$ are the satellites' positions, $\bf{r}_{0}$ is the location
of the central galaxy $\bf{1}$ is the unit matrix, and  
${\bf r}^T$ is the transposed vector $\bf{r}$. 
We use $\bf{r}_0$ as the position of the central galaxy, and not the
satellites' geometrical center, to allow for a fair comparison once
the angular positions of the satellites are randomized around this
point. 

From this tensor we compute its eigenvalues,
$\lambda_1>\lambda_2>\lambda_3$, and corresponding eigenvectors,
$\hat{I}_1$, $\hat{I}_2$, $\hat{I}_3$.
We define the size of the three ellipsoidal axis as
$a=\lambda_1$, $b=\lambda_2$ and $c=\lambda_3$.
We also define $\hat{n}\equiv \hat{I}_1$ as the vector perpendicular to the
planar satellite distribution. 
We also define the Root Mean Squared (RMS) plane width $w$ as the
standard deviation of the satellite distances to the plane defined by
the vector $\hat{n}$.    

To summarize we characterize the satellite distribution with the following
quantities obtained from the inertia tensor: 
\begin{itemize}
\item RMS plane width, $w$.
\item $c/a$ axis ratio.
\item $b/a$ axis ratio.
\end{itemize}



\subsection{Characterizing Asphericity}

We compare each satellite distribution against its own spherically
randomized distribution.
We keep fixed the radial position of every satellite
with respect to the central galaxy and then randomize its angular
position. 
We repeat this procedure 1000 times for each satellite distribution
and proceed to measure the quantities mentioned in the previous section:
$w$, $c/a$ and $b/a$.
For each quantity we compute the average and standard deviation from
the 1000 random samples. 
This allows us to build a normalized version of all quantities of
interest by substracting the mean and dividing by the standard
deviation of the randomized samples.
These normalized quantities are used to build the explicit probability
distributions for the scalars describing asphericity.


\subsection{Building an explicit asphericity probability distribution}

After building the normalized variables with the simulated data we
perform a Kolmogorov-Smirnov test with the null hypothesis of belonging
to a normal distribution with mean and standard deviation computed
from the mean and standard deviation estimated from the data.
Although the physical quantities of interest are bound, unlike the
random variables from a normal distribution, we find that the
distributions for the normalized $w$, $c/a$ and $b/a$ are indeed
consistent with gaussian ditributions. 

Based on this result we build a multivariate normal distribution for
the joint distributions of the normalized $w$, $c/a$ and $b/a$:

\begin{equation}
p(X; \mu, \Sigma) = \frac{1}{(2\pi)^{3/2}|\Sigma|^{1/2}}
\exp\left(-\frac{1}{2}(X-\mu)^{T}\Sigma^{-1}(X-\mu)\right), 
\label{eq:multivariate}
\end{equation}
% 
where $X=[w, c/a, b/a]^{T}$ is a vector variable with the normalized
quantities, $\mu$ is the vector mean and the $\Sigma$ is the
covariance matrix.  


We compute the preferred covariance matrix and the mean distribution values
with a jackknife technique. 
That is, out of the $n$ pairs in each simulation, we perfom $n$
different covariance and mean value measurements using only $n-1$ pairs. 
The reported covariance and mean values correspond to the average of
all measurements, the corresponding standard deviation also helps us to
estimate the uncertainty on every reported coefficient.
This compact description allows us to generate samples of size $N$
that are consistent by construction with their parent simulation. 

Finally, we use the generated samples to estimate how
common are the deviations from sphericity that we measure in the
observational data.  
We use a double-tailed test in this comparison, meaning that we
always measure the fraction of points with absolute values larger than the
treshold absolute observed value. 


% referencia posiciones satellites
% http://adsabs.harvard.edu/abs/2013MNRAS.435.1928P

\begin{figure*}
\centering
\includegraphics[width=0.48\textwidth]{input_illustris1dark_obs_M31_n_11.pdf}
\includegraphics[width=0.48\textwidth]{input_illustris1dark_obs_MW_n_11.pdf}
\caption{Scalars describing satellite shape around the LG galaxies for
  a fixed number of satellites $N_s$=11:
  plane width $w$ in physical units, $c/a$ ratio and $b/a$ ratio.
  Left: M31, rigth: MW. The square at the center are observations; the
2D histograms are the contour levels after randomizing $10^4$ the
satellite positions while keeping the radial distribution fixed; the
blue circles are the results from Illustris-1-Dark. For the M31 these
three results (observations, randomized observations and simulations)
seem to be in broad agreement, while for the MW the three results
clearly differ. 
The corresponding plots for Illustris-1 and ELVIS are
in Figures \ref{fig:all_plots_illustris1} and
\ref{fig:all_plots_elvis},
respectively. \label{fig:physical_illustris1dark}}     
\end{figure*}

\begin{figure*}
\centering
\includegraphics[width=0.48\textwidth]{input_illustris1dark_obs_M31_n_11_normed.pdf}
\includegraphics[width=0.48\textwidth]{input_illustris1dark_obs_MW_n_11_normed.pdf}
\caption{
Same layout as Figure \ref{fig:physical_illustris1dark}. 
  Each scalar
  quantity is re-centered and normalized by the mean value and standard
  deviation from the $10^4$ randomizations. 
  The distribution from the randomizations (plots over the diagonal)
  now has by definition a mean of zero and standard deviation of one. 
The results for the width and $c/a$ ratio of MW satellites are more
than two standard deviations away from the mean. 
The corresponding plots for Illustris-1 and ELVIS are
in Figures \ref{fig:all_plots_illustris1} and
\ref{fig:all_plots_elvis},
respectively. \label{fig:normalized_illustris1dark}}     
\end{figure*}




\begin{figure*}
\centering
\includegraphics[width=0.45\textwidth]{gaussian_model_illustris1dark_M31_n_11.pdf}
\includegraphics[width=0.45\textwidth]{gaussian_model_illustris1dark_MW_n_11.pdf}
\caption{
Same layout as Figure \ref{fig:normalized_illustris1dark}. 
This time the distribution comes from the multivariate gaussian model 
with its parameters estimated from the Illustris-1-Dark results.
The corresponding plots for Illustris-1 and ELVIS are
in Figures \ref{fig:all_plots_illustris1} and \ref{fig:all_plots_elvis},
respectively. \label{fig:gaussian_illustris1dark}}     
\end{figure*}


\section{Results}
\label{sec:results}

Table \ref{table:observations} and Table \ref{table:simulations}
summarize the mean values and uncertainties for the plane $w$, $c/a$
ratio and $b/a$ ratio in the observations and simulations, respectively. 
The uncertainty in the observations is computed from the results with
different number of satellites; in the simulations it corresponds to
the standard deviation over different halos.  

The observed widths $w$ are always smaller than its randomized version.
For M31 there is barely a ratio of $0.92$ between observed and
randomized, while for the MW this factor goes down to less than half
$0.48$. 
The observed $c/a$ ratio follows the same extreme trend for
the MW compared to a M31 distribution closer to spherical, 
The $b/a$ ratio is statisically the same between observations and the
spherical randomization. 
This confirms the extreme planar distribution for the MW and the more
spherical distribution for the M31.

In the following subsections we describe in detail the results on the 
distributions for $w$, $c/a$ and $b/a$. 
The plots in the main body of the paper correspond to the
Illustris-1-Dark simulation; the results from the other simulations
are included in the Appendix \ref{appendix:plots}.

\subsection{Plane Width}

Figure \ref{fig:scatter_width} summarizes the results for the width
measurements.
The left panel compares the results for the MW and M31
observations (stars) against its spherically randomized satellites
(circles). 
The most interesting outcome is that the MW plane width is smaller
than $\approx 98\%$ of the planes computed from the randomized distribution,
while the M31 plane width is only slightly smaller than the average of
the distribution.

The middle panel in Figure \ref{fig:scatter_width} compares the
observational result (star) against the measurements of all pairs from the Illustris1-Dark
simulation (circles).
In this case we have a similar result as before. 
The observed MW width is smaller than all the results in the
simulation, there is not a single halo with similar values.
On the other hand, the results for the M31 are entirely consistent
with observations. Most of the halos in the simulation show a width
value similar to M31. 

The right panel in Figure \ref{fig:scatter_width} shows the result for
the normalized width.  
This panel tells the same story as the middle panel. 
The M31 values are typical while the MW is an outlier. 

The added value of using the normalized data is that this data is 
consistent with normal distributions.
Additionally, this is the data used to build the mean values vector and covariance
matrix described in Equation \ref{eq:multivariate}. 
%The added value of the data in this panel is that it is the normalized
%which is consistent with normal distributions.
%This is the data used to build the mean values vector and covariance
%matrix described in Equation \ref{eq:multivariate}. 



\begin{figure*}
\centering
\includegraphics[width=0.32\textwidth]{LG_numbers.pdf}
\includegraphics[width=0.32\textwidth]{M31_numbers.pdf}
\includegraphics[width=0.32\textwidth]{MW_numbers.pdf}
\caption{
Percentage of systems that have asphericity scalars as extreme as the
LG (left), M31 (middle) and MW (right).  
This percentage is computed as a function of satellite number for all
the three simulations (Illustris-1-Dark, Illustris-1 and ELVIS).
The common trend is that at most of $2\%$ of LG-like pairs are expected to
be as extreme as observations, with most of the weight of this
atypicallity falls onto the MW; at most $4\%$ of similar systems have
such an aspherical satellite distribution. 
\label{fig:expected_number}}
\end{figure*}

\subsection{$c/a$ axis ratio}

Figure \ref{fig:scatter_ca_ratio} shows the results for the minor to
major axis ratio. 
The layout is the same as in Figure \ref{fig:scatter_width}.
The results for the $c/a$ ratio follow the same trends as for the
width $w$.

The left panel in Figure \ref{fig:scatter_ca_ratio} shows how the MW
$c/a$ ratio is significantly lower than the measured values for
spherical distributions, and it is smaller than $\approx 98\%$ of the
randomized distributions.   
On the other hand the ratio for M31 is lower than the mean of the
spherical values but still well within its variance.
The middle panel in the same Figure shows the LG compared against the
results in the simulations. 
In this case we find a similar trend as before. 
The MW is atypical and M31 is within the variance from the simulation data.
This time, however, there are two MW-like halos out of the total
of 24 that show an $c/a$ as small as that of the MW.
The right panel shows the normalized results. 
The MW shows a low $c/a$ ratio between two and three
standard deviations away from the mean value of the spherical
distribution; this contrasts with the results for M31 which are close to
$1$ standard deviation away. 


\subsection{$b/a$ axis ratio}

Figure \ref{fig:scatter_ba_ratio} shows the results for the minor to
major axis ratio with the same layout as Figure \ref{fig:scatter_ca_ratio}
In all cases of comparison (against randomized distribution
and simulations) the results for both the MW and M31 are typical. 


\subsection{Fit to a Multivariate Gaussian Distributions}

Figure \ref{fig:correlations_illustrisdm} illustrates the results from
 computing the covariance matrix and mean vector in
 Eq.\ref{eq:multivariate} from the normalized quantities obtained from
 the Illustris-1-Dark simulation.
The distributions in this Figure are computed from $10^6$ points
generated with the multivariate gaussian. 
Similar plots for Illustris-1 and ELVIS are in the Appendix \ref{appendix:plots}.
The values for all the covariance matrices and mean vectors
corresponding to all the simulations are listed in the Appendix \ref{appendix:covariance}.

This nicely summarizes the results we had in the previous
sections. The left hand triangular plot shows how M31 falls into the middle of all 2D
distributions and is always close to the peak and within the $1\sigma$
range. 
The right hand plot clearly places the MW observations outside the
$3\sigma$ range in the joint distributions that involve the width
$w$. 

In both cases the strongest positive correlation is present for the
width and the $c/a$ axis ratio. A weaker correlation is present for
the width and the $b/a$ axis ratio. 



\subsection{Number of Expected LG Systems}

We use the fits to the multivariate gaussian distributions to
compute the expected number of pairs with characteristics similar to
those of the LG.
To do this we generate $10^3$ samples, each sample containing $10^4$
pairs, where each pair member is drawn from the corresponding
multivariate gaussian distribution.  

We consider that a sampled system is similar to the M31/MW galaxy if the
distance of each of its normalized characteristics ($w$, $c/a$, $b/a$) to the
sample mean is equal or larger than the distance of the observational
values to the sample mean.   
That is, we perform a double-tailed test using the observational
values as a threshold. 

Figure \ref{fig:expected_number} summarizes the results from this
experiment. The left panel shows the probability density for the number of M31
systems in a parent sample of $10^4$ pairs, the right panel shows the
results for the MW.
For M31, between $27\%$ and $56\%$ of the pairs have a satellite
distribution as aspherical as the one observed in M31. This fraction drops
dramatically for the MW where only $0.02\%$ to $3\%$ of the satellites
are expected to have as extreme aspherical distributions as the MW.

Considering the joint distribution of M31 and MW we find that at most
$2\%$ of the pairs are expected to be similar to the LG.
In a three dimensional gaussian distribution, having a $1\sigma$,
$2\sigma$ and $3\sigma$ interval corresponds to having respectively $19 \%$, $73 \%$ and
$97 \%$ of the total of points in the distribution.
With this result in mind the $LG$ has the same degree of atypicality
as a $3\sigma$ outlier. 
% FUNDAMENTALS OF MULTIVARIATE STATISTICS, for Imaging and optics,
% Bajorski.
%In [150]: stats.chi2.cdf(l**2,3)
%Out[150]: array([ 0.19874804,  0.73853587,  0.97070911])

Among the three simulations, the results infered from ELVIS data show
the lowest fraction of M31 and MW systems; for Illustris-1-Dark we have
the highest fraction of M31/MW systems. The results from Illustris-1
are in between these two, but closer to ELVIS.

The most probable reason for these trends is the different median mass
for the MW/M31 halos in the pairs from these simulations. 
For instance for the MW halo the median maximum circular velocity is
$\sim 160$ \kms, $\sim 150$ \kms and $\sim 120$ \kms in the ELVIS,
Illustris-1 and Illustris-1-Dark simulations, respectively.
For the M31 halo this median velocity is $\sim 200$\kms both for the
ELVIS and Illustris1 simulations, while for the Illustris-1-Dark it is
$\sim 160$ \kms. 

Another element that influences this trend is the simulated physics.
This is evident in the comparison between Illustris-1 and Illustris-1-Dark.
These two simulations share the same characteristics except for the
inclusion of baryonic physics in Illustris-1.
We find that extreme MW systems are easier to find in the DM only
simulation. 
The results listed in Appendix \ref{appendix:covariance} show that
halos are closer to spherical once baryonic effects are included. 
For instance the mean value for the normalized width goes closer to
zero, from $-0.45\pm 0.04$  to $-0.17\pm0.04$, and the same is true
for the $c/a$ ratio that changes from $-0.61\pm0.04$ in the DM only
simulation to $-0.40\pm 0.04$ in Illustris-1.  
We postpone a detailed quantification of this effect for a future
study. 

\section{Conclusions}\label{sec:conclusions}

In this paper we developed and demostrated a method to quantify the
asphericity of the satellite distribution around the MW and M31.  
In the interest of keeping the method straightforward and robust we
focus on the asphericity estimates for a fixed number of the brightest
satellites around each galaxy. 

The method uses as a reference the spherically randomized data
of the system under study \citep{2017AN....338..854P}.  
To this end, we first measure the width and axis ratios for the satellite
distributions of interest. 
Then, we measure the same quantities for the same set of points after
the spherical randomziation process.
Finally, we renormalize the initial results to the mean value and
standard deviation computed from the randomized data.  

We found that these normalized quantities are well described by
a multivariate gaussian distribution in spite of the original
quantities being bound.
We estimated the mean and covariance of these distributions using
the results of LG pairs coming from three different numerical
simulations (Illustris-1, Illustris-1-Dark and ELVIS). 
Finally, we compared the observational results against the
distributions derived from the simulations

We found that in the best case (Illustris1-Dark) the degree of
asphericity in the observed LG is only expected in $3\pm2$ pairs out
of a sample of $10^4$ isolated pairs. 
This places the LG as a $3\sigma$ outlier.   
The weight to explain this atypical result is not distributed equally
between the MW and M31. 
While M31 presents a fully typical asphericity in the expectations
from LCDM, the MW shows aspherical deviations in plane width and the
major-to-minor axis ratio highly atypical in the framework of LCDM,
confirming the original hint by \cite{2005A&A...431..517K} and more
recent results by \cite{2015ApJ...815...19P}. 
We estimated that with the M31 between $37\%$ and $58\%$ of the pairs show aspherical characteristics larger than M31, while
this fraction drops to less than $1\%$ for the MW. 
These fractions are robust to changes in the numerical simulations, 
the criteria used to define the pairs and to the methods used to
estimate the parameters of the multivariate normal distribution. 


The focus of our approach was building an explicit probability
distribution for the observables of interest, instead of trying to
find simulated objects that fulfill different observational criteria. 
This approach is particularly useful in the case of atypical
observables, as it allows for an atypicallity
quantification without the need to build explicit samples of objects
that are already scarce and difficult to find in simulations.

An extension of this framework to outliers in higher order deviations
(i.e. coherent \emph{velocity} structures) should also be possible, 
provided that an explicit probability distribution for the scalars of
interest can be built.  
The approach presented here is also useful to gauge the influence of 
influence of different physical elements includes the simulation. 
For instance, in our case the data hints towards rounder satellite
distributions in simulations that include baryonic effects.   


The highly aspherical satellite distribution in the MW is another piece of
information that points at an atypical configuration in LCDM.
We also have the number of satellites as bright as the Mageallanic
Clouds, only expected in $5\%$ of galaxies
\citep{2011ApJ...743..117B} and 
the satellite velocities around the MW with a radial/tangential
anisotropy only expected in $3\%$ of systems in LCDM
\citep{2017MNRAS.468L..41C}. 
One could also add the atypical kinematics of M31 with a very low
tangential velocity, which is only expected in less than $1\%$ of the pairs
with similar environmental characteristics \citep{ForeroRomero2013}.  

This atypicality should be seen as an opportunity to constrain in
great detail the environment that allowed such a pattern to emerge. 
Although broad correlations between LG assembly, pair kinematics, halo
shapes and satellite distributions are expected in LCDM
\citep{2011MNRAS.417.1434F,2014MNRAS.443.1090F,2015ApJ...799...45F,2015MNRAS.452.1052L},
we claim that detailed studies of the satellite asphericity as a function
of halo mass and cosmic web environment are still missing to
understand what features in the initial conditions of our LG
are responsible for the extreme features observed in its satellite
distribution.

\section*{Acknowledgements} 
We acknowledge financial support from: Universidad de Los Andes
(Colombia); COLCIENCIAS (c\'odigo 120471250459, contrato FP44842-287-
2016); the European Union's Horizon 2020 Research and Innovation
Programme under the Marie Sk\l{}odowska-Curie grant agreement No 73437
(LACEGAL).  
  
\bibliographystyle{mnras}
\bibliography{Dwarfs}

\appendix


\section{Physical Characteristics of the Isolated Pairs samples}
\label{appendix:physical}

\begin{figure}
\centering
\includegraphics[width=0.30\textwidth]{int_distro_LG_v_rad.pdf}
\includegraphics[width=0.30\textwidth]{int_distro_LG_d.pdf}
\includegraphics[width=0.30\textwidth]{int_distro_M31_vmax.pdf}
\includegraphics[width=0.30\textwidth]{int_distro_MW_vmax.pdf}
\caption{Physical characteristics of the LG pairs selected in the
  simulations. All plots show the integrated distributions. The
  physical properties are the radial comoving velocity between the MW
  and M31, the radial separation between the MW and M31 and the
  maximum circular velocity for the M31 and MW dark matter halo.
\label{fig:physical_pairs}}
\end{figure}



\section{Covariance Matrices and Mean value vectors}
\label{appendix:covariance}
\subsection{Illustris-1-Dark}

\subsubsection{M31}
\[
\Sigma=
\begin{bmatrix}
2.02 \pm 0.09 & 1.46 \pm 0.09 & 0.97 \pm 0.07\\
1.46 \pm 0.09 & 1.69 \pm 0.10 & -0.01 \pm 0.05\\
0.97 \pm 0.07 & -0.01 \pm 0.05 & 1.42 \pm 0.08\\
\end{bmatrix}
\]

\[
\mu=
\begin{bmatrix}
-0.43 \pm 0.06 & -0.56 \pm 0.06 & -0.17 \pm 0.05\\
\end{bmatrix}
\]

\subsubsection{MW}

\[
\Sigma=
\begin{bmatrix}
0.94 \pm 0.04 & 0.71 \pm 0.04 & 0.36 \pm 0.03\\
0.71 \pm 0.04 & 0.93 \pm 0.05 & -0.27 \pm 0.03\\
0.36 \pm 0.03 & -0.27 \pm 0.03 & 0.94 \pm 0.04\\
\end{bmatrix}
\]

\[
\mu=
\begin{bmatrix}
-0.40 \pm 0.04 & -0.69 \pm 0.04 & -0.19 \pm 0.04\\
\end{bmatrix}
\]


\subsection{Illustris-1}
\subsubsection{M31}
\[
\Sigma=
\begin{bmatrix}
1.59 \pm 0.08 & 1.23 \pm 0.07 & 0.80 \pm 0.06\\
1.23 \pm 0.07 & 1.44 \pm 0.08 & 0.01 \pm 0.06\\
0.80 \pm 0.06 & 0.01 \pm 0.06 & 1.26 \pm 0.08\\
\end{bmatrix}
\]
\[
\mu=
\begin{bmatrix}
0.03 \pm 0.06 & -0.24 \pm 0.06 & -0.01 \pm 0.05\\
\end{bmatrix}
\]
\subsubsection{MW}
\[
\Sigma=
\begin{bmatrix}
1.01 \pm 0.05 & 0.72 \pm 0.05 & 0.62 \pm 0.04\\
0.72 \pm 0.05 & 0.78 \pm 0.05 & 0.12 \pm 0.03\\
0.62 \pm 0.04 & 0.12 \pm 0.03 & 0.80 \pm 0.04\\
\end{bmatrix}
\]
\[
\mu=
\begin{bmatrix}
-0.07 \pm 0.05 & -0.45 \pm 0.04 & 0.02 \pm 0.04\\
\end{bmatrix}
\]

\subsection{ELVIS}

\subsubsection{M31}
\[
\Sigma=
\begin{bmatrix}
1.77 \pm 0.23 & 1.44 \pm 0.18 & 0.65 \pm 0.09\\
1.44 \pm 0.18 & 1.54 \pm 0.15 & 0.15 \pm 0.06\\
0.65 \pm 0.09 & 0.15 \pm 0.06 & 0.77 \pm 0.11\\
\end{bmatrix}
\]
\[
\mu=
\begin{bmatrix}
0.07 \pm 0.09 & -0.06 \pm 0.08 & -0.08 \pm 0.06\\
\end{bmatrix}
\]
\subsubsection{MW}
\[
\Sigma=
\begin{bmatrix}
0.59 \pm 0.06 & 0.61 \pm 0.11 & 0.11 \pm 0.08\\
0.61 \pm 0.11 & 1.25 \pm 0.19 & -0.71 \pm 0.12\\
0.11 \pm 0.08 & -0.71 \pm 0.12 & 1.13 \pm 0.11\\
\end{bmatrix}
\]
\[
\mu=
\begin{bmatrix}
-0.25 \pm 0.05 & -0.59 \pm 0.08 & -0.16 \pm 0.07\\
\end{bmatrix}
\]


\section{Results from ELVIS and Illustris1}
\label{appendix:plots}
\begin{figure*}
\centering
\includegraphics[width=0.40\textwidth]{input_illustris1_obs_M31_n_11.pdf}
\includegraphics[width=0.40\textwidth]{input_illustris1_obs_MW_n_11.pdf}
\includegraphics[width=0.40\textwidth]{input_illustris1_obs_M31_n_11_normed.pdf}
\includegraphics[width=0.40\textwidth]{input_illustris1_obs_MW_n_11_normed.pdf}
\includegraphics[width=0.40\textwidth]{gaussian_model_illustris1_M31_n_11.pdf}
\includegraphics[width=0.40\textwidth]{gaussian_model_illustris1_MW_n_11.pdf}
\caption{Illustris-1 results for the quantities presented for the Illustris1-Dark
  simulation in Figures  \ref{fig:scatter_width},
  \ref{fig:scatter_ca_ratio}, \ref{fig:scatter_ba_ratio}.
Upper row corresponds to the raw values from observations and
simulated pairs, while the second row normalizes the same values to
the mean and standard deviation on its spherically randomized
counterparts. 
\label{fig:all_plots_illustris1}}
\end{figure*}


\begin{figure*}
\centering
\includegraphics[width=0.40\textwidth]{input_elvis_obs_M31_n_11.pdf}
\includegraphics[width=0.40\textwidth]{input_elvis_obs_MW_n_11.pdf}
\includegraphics[width=0.40\textwidth]{input_elvis_obs_M31_n_11_normed.pdf}
\includegraphics[width=0.40\textwidth]{input_elvis_obs_MW_n_11_normed.pdf}
\includegraphics[width=0.40\textwidth]{gaussian_model_elvis_M31_n_11.pdf}
\includegraphics[width=0.40\textwidth]{gaussian_model_elvis_MW_n_11.pdf}
\caption{ELVIs results for the quantities presented for the Illustris1-Dark
  simulation in Figures  \ref{fig:scatter_width},
  \ref{fig:scatter_ca_ratio}, \ref{fig:scatter_ba_ratio}.
Upper row corresponds to the raw values from observations and
simulated pairs, while the second row normalizes the same values to
the mean and standard deviation on its spherically randomized
counterparts. 
\label{fig:all_plots_elvis}}
\end{figure*}


\begin{figure*}
\centering
\includegraphics[width=0.40\textwidth]{gaussian_model_illustris1dark_M31_n_12.pdf}
\includegraphics[width=0.40\textwidth]{gaussian_model_illustris1dark_MW_n_12.pdf}
\includegraphics[width=0.40\textwidth]{gaussian_model_illustris1_M31_n_12.pdf}
\includegraphics[width=0.40\textwidth]{gaussian_model_illustris1_MW_n_12.pdf}
\includegraphics[width=0.40\textwidth]{gaussian_model_elvis_M31_n_12.pdf}
\includegraphics[width=0.40\textwidth]{gaussian_model_elvis_MW_n_12.pdf}
\caption{ELVIs results for the quantities presented for the Illustris1-Dark
  simulation in Figures  \ref{fig:scatter_width},
  \ref{fig:scatter_ca_ratio}, \ref{fig:scatter_ba_ratio}.
Upper row corresponds to the raw values from observations and
simulated pairs, while the second row normalizes the same values to
the mean and standard deviation on its spherically randomized
counterparts. 
\label{fig:scatter_elvis}}
\end{figure*}




\end{document}
